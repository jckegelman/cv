\cvsection{Experience}
\begin{cventries}
  \cventry%
    {Algorithms and Vehicle Dynamics Lead}
    {Peloton Technology}
    {Mountain View, CA}
    {Aug.\ 2019--PRESENT}
    {%
      \begin{cvitems}
        \item{Led team of four vehicle software engineers developing control, estimation, and perception algorithms.}
        \item{Helped establish a BigTable database to facilitate access to and analysis of logged vehicle data using C++ and Python tools.}
        \item{Built and released three major, three minor, and seventeen patch versions of Peloton's vehicle software.}
        \item{Developed architecture for integrating with automotive radar sensors from two different manufacturers.}
      \end{cvitems}
    }

  \cventry%
    {Senior Software Engineer}
    {Peloton Technology}
    {Mountain View, CA}
    {May 2017--Aug.\ 2019}
    {%
      \begin{cvitems}
        \item{Developed safety-critical, production software in a continuous integration and testing environment.}
        \item{Implemented a distributed safety monitoring system in C++ for a commercial platooning system following the ISO 26262 standard.}
        \item{Collected, compiled, and analyzed braking data to inform the safety of the intended functionality (SOTIF) analysis of a commercial platooning system.}
        \item{Developed a graphical user interface in Python using Matplotlib and pandas for vehicle data visualization and exploratory data analysis.}
        \item{Contributed to Peloton's estimation, modeling, and control modules, including comprehensive simulation, software-in-the-loop, and hardware-in-the-loop testing environments.}
        \item{Incorporated automatic Python linting (pylint) and formatting (YAPF) into Peloton's build and test infrastructure (Bazel run by Buildbot).}
      \end{cvitems}
    }

  \cventry%
    {Graduate Research Assistant, PI:\ Prof.~J.~Christian Gerdes}
    {Dynamic Design Lab}
    {Stanford, CA}
    {Sep.\ 2009--Dec.\ 2018}
    {%
      \begin{cvitems}
        \item{Collected, compiled, analyzed, and openly published vehicle dynamics data from highly-skilled professional race car drivers during live racing events to gain insights into vehicle control at the limits of handling. Compared human performance with autonomous vehicles to improve operating capabilities of active vehicle safety systems.}
        \item{Designed and built a comprehensive, noninvasive vehicle instrumentation and data acquisition system for vintage race cars with significant historical value.}
        \item{Developed a graphical user interface in MATLAB for vehicle data visualization and exploratory data analysis.}
        \item{Implemented autonomous vehicle control using drive-by-wire hardware and convex optimization software to operate at the handling limits while following a desired trajectory. Implemented and tested control algorithms on experimental vehicles using C and MATLAB.}
        \item{Assembled and maintained an end-to-end solution from surveyed GNSS base stations to on-board integrated navigation systems enabling research vehicles to operate reliably with centimeter-level position measurement accuracy. Installed and operated a Linux-based NTRIP caster to broadcast Differential GNSS corrections from multiple servers to multiple clients.}
      \end{cvitems}
    }

  \cventry%
    {Senior Teaching Assistant --- Mechanical Systems Design, Profs.~Mark Cutkosky and~Paul Mitiguy}
    {Stanford Department of Mechanical Engineering}
    {Stanford, CA}
    {Jan.--Mar.\ 2016}
    {%
      \begin{cvitems}
        \item{Developed curriculum and coordinated team of five other teaching assistants.}
        \item{Led hands-on laboratory and tutorial sessions for course with 150 upperclassmen and co-term Mechanical Engineering and Product Design students exploring characteristics of machine elements.}
        \item{Advised design-project teams emphasizing the balance of physical and virtual prototyping based on engineering analysis.}
      \end{cvitems}
    }
\end{cventries}
